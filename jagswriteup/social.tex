\documentclass{article}
\usepackage[margin = 1.0in]{geometry} %all around 1 inch margins
\usepackage{booktabs} %nice looking tables
\usepackage{threeparttable} %nice footnote for tables
\usepackage{graphicx}
%\usepackage[section]{placeins} %keeps floats from floating over a section
\usepackage{rotating} %rotate floats
\usepackage{amsmath}
\usepackage[table,xcdraw]{xcolor}
\usepackage{listings}
\lstset{
    breaklines=true,
%    postbreak=\raisebox{0ex}[0ex][0ex]{\ensuremath{\color{red}\hookrightarrow\space}}
}
\makeatletter% Set distance from top of page to first float
\setlength{\@fptop}{5pt}
\makeatother
% \usepackage{epstopdf}


\begin{document}
\noindent 14 Mar 2015

\vspace*{0.25cm}

In the \underline{Analyzing Animal Societies} book (Whitehead 2008) chapter 9.4 (appendices) the method for calculating social differentiation in socprog is explained. Basically, social differentiation $S$ is defined as the coefficient of variation of the association indices, $\alpha_{ij}$, for individuals $ij$ where $i \ne j$. The coefficient of variation $CV$ is defined as the ratio of the standard deviation to the mean:

\[CV = \frac{\sigma}{\mu}\]

Our association indices are just estimates calculated by the number of times individuals $ij$ are seen together, $x_{ij}$, divided by the total number of samples for $i$ and $j$, $d_{ij}$. We can think of $\alpha_{ij}$ as the true unobserved association index and $x_{ij}$ a random variable binomially distributed.

\[x_{ij} \sim  Binom(d_{ij}, \alpha_{ij})\]

The association indices are also treated as random variables ranging between 0 and 1 drawn from the beta distribution.

\[\alpha_{ij} \sim Beta(u_1,u_2)\]

The $u_1$ and $u_2$ are the beta distribution parameters. They can be solved in terms of the mean, $\mu$, and standard deviation, $\sigma$ from which we can easily calculate the social differentiation coefficient $S$.

\begin{align*}
	u_1 &= \mu \cdot \left(\frac{1-\mu}{\mu \cdot (\frac{\sigma}{\mu})^2} - 1\right) \\
	u_2 &= (1-\mu) \cdot \left(\frac{1-\mu}{\mu \cdot (\frac{\sigma}{\mu})^2} -1\right) \\
\end{align*}

Socprog uses a maximum likelihood monte carlo approach to choose $\mu$ and $\sigma$, by maximizing:

\[Likelihood = Binom(x_{ij}; d_{ij}, \alpha_{ij}) \cdot Beta(\alpha_{ij}, u_1, u_2)\]

The likelihood is written slightly differently in Whitehead (2008), but it is equivalent. To make this a posterior distribution instead of a likelihood, I added priors for $\mu$ and $\sigma$. Both of these priors are truncated normal, centered on the estimated $\mu_{est}$ and $\sigma_{est}$ calculated from $x_{ij}$ and $d_{ij}$ and a large variance of 1000. Both distributions are truncated at 0, since negative values don't make sense in the context of association indices.

\begin{align*}
	\mu &\sim N(\mu_{est}, 1000);~ 0 < \mu < \infty\\
	\sigma &\sim N(\sigma_{est}, 1000); ~ 0 < \sigma < \infty\\
\end{align*}

I used JAGS (just another gibbs sampler) and R to estimate $\mu$ and $\sigma$.

\end{document}
